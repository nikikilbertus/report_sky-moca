%
%%%%%%%%%%%%%%%%%%%%%%%%%%%%%%%%%%%%%%%%%%%%%%%%%%%%%%%%%%%%%%%%%%%%%%%%%%%%%%%%
\chapter{Theoretical background}\label{chap:1}
%%%%%%%%%%%%%%%%%%%%%%%%%%%%%%%%%%%%%%%%%%%%%%%%%%%%%%%%%%%%%%%%%%%%%%%%%%%%%%%%
%
%%%%%%%%%%%%%%%%%%%%%%%%%%%%%%%%%%%%%%%%%%%%%%%%%%%%%%%%%%%%%%%%%%%%%%%%%%%%%%%%
\section{Introduction}\label{sec:intro}
%%%%%%%%%%%%%%%%%%%%%%%%%%%%%%%%%%%%%%%%%%%%%%%%%%%%%%%%%%%%%%%%%%%%%%%%%%%%%%%%
%
Since the \todo{theoretical prediction?} of skyrmions in \todo{year?}
\todo{cite} and their experimental discovery in \todo{year} \todo{cite},
skyrmions have been extensively investigated both by theorists and
experimentalists \todo{some refs}. Skyrmions have caused a lot of excitement
mostly due to their promising properties that might qualify them as fast and
efficient memory. Three-dimensional non-perturbative classical Monte Carlo
methods developed recently~\cite{skyrmionlattice}, allow us to compute the full
finite temperature phase diagrom and explore phase transitions. The goal of this
project was to implement the discretized lattice hamiltonian introduced
in~\cite{skyrmionlattice} and basically reproduce the results discussed there.
In contrast to most previous work on Monte Carlo methods describing skyrmions,
we want to provide a detailed exposition of the algorithms and numerical
aspects. The report can serve as an ab-initio step-by-step introdcution on how
to write Monte Carlo methods and associated pitfalls or caveats based on a
concrete example. All code is publicly available on GitHub under
\todo{link} and we hope that it will be useful to others.

We introduce the model in \secref{sec:model} and briefly outline the genearl
idea of Monte Carlo methods on an abstract level. In \secref{sec:mctheory} we
introduce Monte Carlo techniques more formally and state some important
properties and results mostly without proofs. Moreover we discuss verification
strategies for Monte Carlo codes. \Secref{sec:code} contains the documentation
of our code as well as a user manual. Moreover, we provide reasoning for design
decisions. Physical results of the simulations are presented in
\secref{sec:results}. Finally, we conclude in \chapref{chap:3}.
%
%%%%%%%%%%%%%%%%%%%%%%%%%%%%%%%%%%%%%%%%%%%%%%%%%%%%%%%%%%%%%%%%%%%%%%%%%%%%%%%%
\section{The Spin Lattice Model}\label{sec:model}
%%%%%%%%%%%%%%%%%%%%%%%%%%%%%%%%%%%%%%%%%%%%%%%%%%%%%%%%%%%%%%%%%%%%%%%%%%%%%%%%
%
A common high-level way to think about magnetism in condensed matter is in terms
of complex collective behavior of spins, which are associated to a magnetic
moment. Astoundingly this simple model is quite powerful. The model we use here
consists of a three-dimensional lattice with equidistant spacing in each
direction. To each lattice site we attach a spin, represented as an arbitrary
element of the unit sphere~$S^2$, see \figref{fig:lattice}.

% \tdplotsetmaincoords{60}{110}
\begin{figure}
\centering
\begin{tikzpicture}%[tdplot_main_coords]
  \lattice{3}
  % \latticeflat{4}
\end{tikzpicture}%
\caption{A cubic three-dimensional lattice with a spin at each lattice site.}
\label{fig:lattice}
\end{figure}
%
%%%%%%%%%%%%%%%%%%%%%%%%%%%%%%%%%%%%%%%%%%%%%%%%%%%%%%%%%%%%%%%%%%%%%%%%%%%%%%%%
\section{Monte Carlo Methods}\label{sec:mctheory}
%%%%%%%%%%%%%%%%%%%%%%%%%%%%%%%%%%%%%%%%%%%%%%%%%%%%%%%%%%%%%%%%%%%%%%%%%%%%%%%%
%
\todo{}
