%
%%%%%%%%%%%%%%%%%%%%%%%%%%%%%%%%%%%%%%%%%%%%%%%%%%%%%%%%%%%%%%%%%%%%%%%%%%%%%%%%
\chapter{Theoretical background}\label{chap:1}
%%%%%%%%%%%%%%%%%%%%%%%%%%%%%%%%%%%%%%%%%%%%%%%%%%%%%%%%%%%%%%%%%%%%%%%%%%%%%%%%
%
%%%%%%%%%%%%%%%%%%%%%%%%%%%%%%%%%%%%%%%%%%%%%%%%%%%%%%%%%%%%%%%%%%%%%%%%%%%%%%%%
\section{Introduction}\label{sec:intro}
%%%%%%%%%%%%%%%%%%%%%%%%%%%%%%%%%%%%%%%%%%%%%%%%%%%%%%%%%%%%%%%%%%%%%%%%%%%%%%%%
%
Since the \todo{theoretical prediction?} of skyrmions in \todo{year?}
\todo{cite} and their experimental discovery in \todo{year} \todo{cite},
skyrmions have been extensively investigated both by theorists and
experimentalists \todo{some refs}. Skyrmions have caused a lot of excitement
mostly due to their promising properties that might qualify them as fast and
efficient memory. Three-dimensional non-perturbative classical Monte Carlo
methods developed recently~\cite{skyrmionlattice}, allow us to compute the full
finite temperature phase diagrom and explore phase transitions. The goal of this
project was to implement the discretized lattice hamiltonian introduced
in~\cite{skyrmionlattice} and basically reproduce the results discussed there.
In contrast to most previous work on Monte Carlo methods describing skyrmions,
we want to provide a detailed exposition of the algorithms and numerical
aspects. The report can serve as an ab-initio step-by-step introdcution on how
to write Monte Carlo methods and associated pitfalls or caveats based on a
concrete example. All code is publicly available on GitHub under
\todo{link} and we hope that it will be useful to others.

We introduce the model in \secref{sec:model} and briefly outline the genearl
idea of Monte Carlo methods on an abstract level. In \secref{sec:mctheory} we
introduce Monte Carlo techniques more formally and state some important
properties and results mostly without proofs. Moreover we discuss verification
strategies for Monte Carlo codes. \Secref{sec:code} contains the documentation
of our code as well as a user manual. Moreover, we provide reasoning for design
decisions. Physical results of the simulations are presented in
\secref{sec:results}. Finally, we conclude in \chapref{chap:3}.
%
%%%%%%%%%%%%%%%%%%%%%%%%%%%%%%%%%%%%%%%%%%%%%%%%%%%%%%%%%%%%%%%%%%%%%%%%%%%%%%%%
\section{The Spin Lattice Model}\label{sec:model}
%%%%%%%%%%%%%%%%%%%%%%%%%%%%%%%%%%%%%%%%%%%%%%%%%%%%%%%%%%%%%%%%%%%%%%%%%%%%%%%%
%
A common high-level way to think about magnetism in condensed matter is in terms
of complex collective behavior of spins, each of which is associated to a
magnetic moment. Astoundingly this simple model is quite powerful and allows for
an thorough explanation of a wide range of phenomena. Let us consier a
three-dimensional lattice with equidistant spacing in each direction. To each
lattice site we attach a spin, represented by an arbitrary element of the unit
sphere~$S^2$, see \figref{fig:lattice}. In the following we will often resort to
the two-dimensional model for illustration purposes, because it is easier to
draw on a two-dimensional surface. Each lattice sites typically represents a
nucleus, hence the whole lattice can be interpreted as a regular atomic
structure of a solid.

We work with a simple cubic lattice
%
\begin{equation}
  \Sigma := \numlist{1}{N_x} \times \numlist{1}{N_y} \times
  \numlist{1}{N_z} \subset \bN^3 \subset \bR^3\:,
\end{equation}
%
where we interpret~$(i,j,k) \in \Sigma$ as~$i \hat{\x} + j \hat{\y} + k \hat{\z}
\in \bR^3$. Here,~$\hat{\x}, \hat{\y}, \hat{\z}$ are the standard basis vectors
of~$\bR^3$. At each point~$\r \in \Sigma$ we attach a spin~$\S_{\r} \in S^2$,
which yields an overall configuration space~$\Pi := \prod_{\r \in \Sigma} S^2$.
Note that we only discretize the positions of the spins, but not explicitly
their directions.  In any real implementation they are of course discretized by
the finite count of floating point numbers. This mirrors the naturally
discretized crystal structure of solids.

% While the microscopic field within the solid varies greatly over distances on
% the order of the lattice spacing, from far away those microscopic fluctuations
% smear out and we observe an averaged macroscopic field. For example, when we
% choose each spin uniformly at random, as we did in \figref{fig:lattice}, one
% expects the overall macroscopic field to be zero for a sufficiently large
% system.

The goal of our simulation is to find a state~$\pi \in \Pi$ that minimizes the
energy of the system for given parameters. You might notice that we have not
defined a notion of energy yet nor explicitly listed the free parameters of such
a spin lattice. Apparently any physical measure of energy must be based on
interactions between the spins within the system or also with external fields.
Let us elaborate on some possible interactions. Each of them comes with a
constant that can be interpreted as a weight, \ie{} how much the respective
interaction contributes to the action with respect to the others. Those
constants are parameters of the system.

\subsection{Interactions}

The most obvious interaction is the \newterm{ferromagnetic} or \newterm{direct}
exchange.  Pictorially speaking, it favors constellations where spins that are
close to each other point into the same direction. A system only interacting
this way will end up in a state where all spins are parallel to each other. The
measure for parallelism of two neighbouring spins~$\S_{\r_1}, \S_{\r_2} \in S^2$
can be expressed as~$- \S_1 \cdot \S_2 = - \norm{\S_1} \norm{\S_2}
\cos(\alpha)$, where~$\alpha$ is the angle between~$\S_1$ and~$\S_2$. The minus
sign ensures that the energy of two parallel spins is smaller than the energy of
two perpendicular or even antiparallel ones. In the continuum the direct
exchange term consists of a gradient, which is a local quantity, \ie{} the
gradient at a point only depends on an arbitrarily small neighbourhood of the
point. Thus we will always consider the ferromagnetic exchange to be
\newterm{local} or \newterm{short ranged} in the sense that it only contributes
for adjacent lattice sites and we can thus hide the distance in a leading
constant, see \todo{figref}.

Another important exchange term describes the interaction of the system with an
\newterm{external magnetic field}. Clearly, every spin tries to align with an
external field~$\B$, which we express mathematically via~$-\B \cdot \S$ for
every spin~$\S$ on the lattice. It can be misleading to use terms such as
\newterm{non-local} or \newterm{long ranged} for this exchange, since it is not
an interaction between two or more spins within the system, but affects each
lattice site independently in the same way.

These two are the most commonly encountered exchange terms. The
\newterm{dipole-dipole interaction} is somewhat more complex, but also weaker.
For two spins~$\S_1, \S_2 \in S^2$ at positions~$\r_1, \r_2 \in \bR^3$, it is
given by
%
\begin{equation}
  - \frac{1}{\norm{\r}^3}
  (3 (\S_1 \cdot \hat{\r})(\S_2 \cdot \hat{\r}) - \S_1 \cdot \S_2)\:,
\end{equation}
%
where~$\r = \r_2 - \r_1$ and~$\hat{\r} = \r / \norm{\r}$. The dipole-dipole
interaction depends on the distance between the two lattice sites as well as the
orientation of the two spins not only relative to each other, but also to the
line connecting them. Moreover, the explicit dependense on the relative position
already indicates that the dipole-dipole interaciton is relevant for each pair
of spin in the system, it is a long ranged interaction. In a system of~$N^3$
lattice sites, the number of pairs scales like~$N^6$. Hence, most simulations do
not take the dipole-dipole exchange into account purely due to limited
computational resources. The same holds true for our example.

In this work we are interested in chrial magnets, certain crystals that lack
inversion symmetry, \eg{} MnSi. This gives rise to the so called \newterm{weak
Dzyaloshinskii-Moriya} (DM) coupling. In the continuum it is described by a term
proportional to~$-\S(\r) \cdot (\nabla \times \S(\r)$. Just like the gradient,
the curl of a vector field is a local property, thus the DM exchange only
contributes for adjacent lattice sites. The discretized version for two
spins~$\S_1, \S_2 \in S^2$ at neighbouring positions~$\r_1, \r_2 \in \bR^3$
\todo{lattice instead of R3} reads~$- (\S_1 \times \S_2) \cdot (\r_2 - \r_1)$.
Since the cross product is zero for parallel vectors and maximal for
perpendicular vector, the DM coupling acts to some extent against the
ferromagnetic interaction and favors constellations where adjacent spins are
perpendicular to each other.

\todo{Why is ferromagnetic exchange always canonically local (neighbouring only)
and dipole dipole interaction would be between all pairs?}


\begin{figure}
  \centering
  \begin{tikzpicture}
    \latticeflat{4}
    \begin{scope}[xshift=8cm,yshift=1cm]
      \lattice{4}
    \end{scope}
  \end{tikzpicture}%
  \caption{On the left side we show a two-dimensional spin lattice. Each lattice
  site represents a magnetic moment or spin, represented by an arrow of unit
  length, \ie{}, in the two-dimensional picture, by an element of~$S^1$. The
  three-dimensional picture on the right side becomes unclear in a
  two-dimensional drawing rather quickly. Note that the magnetic moments are now
  also three-dimensional, \ie{} elements of the two-dimensional unit
  sphere~$S^2$.}
\label{fig:lattice}
\end{figure}

\begin{figure}
  \centering
  \begin{tikzpicture}
    \latticeselect{3}
    \begin{scope}[xshift=5cm]
      \latticeinter{3}
    \end{scope}
    \begin{scope}[xshift=10cm, yshift=-1cm]
      \latticeinternn{5}
    \end{scope}
  \end{tikzpicture}%
  \caption{\todo{}}
\label{fig:interact}
\end{figure}
%
%%%%%%%%%%%%%%%%%%%%%%%%%%%%%%%%%%%%%%%%%%%%%%%%%%%%%%%%%%%%%%%%%%%%%%%%%%%%%%%%
\section{Monte Carlo Methods}\label{sec:mctheory}
%%%%%%%%%%%%%%%%%%%%%%%%%%%%%%%%%%%%%%%%%%%%%%%%%%%%%%%%%%%%%%%%%%%%%%%%%%%%%%%%
%
\todo{}
